% -----------------------------------------------
% chktex-file 44
\documentclass[../index.tex]{subfiles}

% -----------------------------------------------

\begin{document}

% -----------------------------------------------
\renewcommand{\currenttitle}{Here's a list}
\begin{frame}{\currenttitle}
  \onslide<+->{\texttt{[0, 1, 2, 3, 4, 5]}}
  \onslide<+->{$\rightarrow$ \texttt{[0, 1, 4, 9, 16, 25]}}
\end{frame}

% ---------------------------
\renewcommand{\currenttitle}{A for loop solution}
\begin{frame}[fragile]{\currenttitle}
  Some Python code for transforming this list:

  \begin{lstlisting}[language=Python]<1->
def square(n):
  return n * n

list      = [0, 1, 2, 3, 4, 5]
new_list  = [0, 0, 0, 0, 0, 0]

for i in range(0, len(list)):
  new_list[i] = square(list[i])

print(new_list)
# [0, 1, 4, 9, 16, 25]
  \end{lstlisting}

  \onslide<2->{\dots but this is kind of verbose}
\end{frame}

% ---------------------------
\renewcommand{\currenttitle}{Using \texttt{map}}
\begin{frame}[fragile]{\currenttitle}
  We can use a concept called \texttt{map} to do the same thing in a more
  readable way: \\[1em]

  \begin{lstlisting}[language=Python]<1->
def square(n):
  return n * n

numbers = [0, 1, 2, 3, 4, 5]
new_numbers = map(square, numbers)

print(new_numbers)
# [0, 1, 4, 9, 16, 25]
  \end{lstlisting}

  \onslide<2->{\dots what just happened?}
\end{frame}

% ---------------------------
\renewcommand{\currenttitle}{What is \texttt{map}}
\begin{frame}[fragile]{\currenttitle}
  \onslide<+->{\texttt{map} \dots}

  \begin{enumerate}
    \item<+-> takes an iterable container (i.e. a list) and a function
    \item<+-> applies the function to each element of the container
    \item<+-> returns a new container with the resulting elements
  \end{enumerate}
\end{frame}

% ---------------------------
\renewcommand{\currenttitle}{Iterators 101}
\begin{frame}[fragile]{\currenttitle}
  Think of iterators as objects that yield, on each iteration, the next value
  in a collection (i.e. a list)

  In imperative languages that use the iterator concept, it may be best
  understand like so (this is pseudocode): \\[1em]

  \begin{lstlisting}[language=Python]
# We have a list
numbers = [1, 2, 4]
# Let's get an iterator on that list
iter = iterator(numbers)
# We can now iterate through the values like so
v = next(iter)  # v == 1
v = next(iter)  # v == 2
v = next(iter)  # v == 4
v = next(iter)  # v == null / None  
                #   ^ We know the iterator is done
  \end{lstlisting}
\end{frame}

% ---------------------------
\begin{frame}[fragile]{\currenttitle}
  Iterators are \textbf{lazily evaluated}; the next value isn't computed /
  loaded until we call \texttt{next}.

  The syntax \texttt{1..} in Rust produces an infinite iterator:

  \begin{lstlisting}[language=Python]
    iter = 1..
    v = next(iter)    # v == 1
    v = next(iter)    # v == 2
    v = next(iter)    # v == 3
                      # ... and forever
  \end{lstlisting}

  We can't have an infinite list (we don't have infinite memory), but because of
  lazily evaluation, we've only loaded the values 1, 2, and 3. 

  4, 5, 6, \ldots aren't loaded.
\end{frame}

% ---------------------------
\begin{frame}[fragile]{\currenttitle}
  For the sake of demonstrating concepts:

  \begin{itemize}
    \item \texttt{iterator(list)} creates an iterator from a collection (such as a
          list)
    \item \texttt{next(iterator)} gets the next value of an iterator
    \item \texttt{list(iterator)} collects an iterator back into a list
          (effectively call \texttt{next} until we've gotten all the values and put
          them in a list)
  \end{itemize}

  \vspace*{2em}
  The syntax is modeled after Python for familiarity, but the following code is
  not valid Python.
\end{frame}

% ---------------------------
\renewcommand{\currenttitle}{Pseudo-implementation of \texttt{map}}
\begin{frame}[fragile]{\currenttitle}
  Thus, \texttt{map}'s implementation might look something like this: \\[1em]

  \begin{lstlisting}[language=Python]
# Effectively returns another iterator
def map(func, iter):
  while true:
    elem = next(iter)
    if elem == null:
      break
    # Yield as the 'next' value for the iterator
    yield func(elem)

iter1 = iterator([1, 2, 3, 4])
iter2 = map(square, iter1)
v = next(iter2)   # v == 1
v = next(iter2)   # v == 4
v = next(iter2)   # v == 9
  \end{lstlisting}
\end{frame}

% ---------------------------
\renewcommand{\currenttitle}{Another \texttt{map} example}
\begin{frame}[fragile]{\currenttitle}
  Here's a \textit{slightly} more complex map example:

  \begin{lstlisting}[language=Python]<1->
def transform(n):
  if n % == 0:
    return 'even'
  else:
    return 'odd'

iter = iterator([0, 1, 2, 3, 4, 5])
mapped = map(transform, iter)

print(list(mapped))
# ['even', 'odd', 'even', 'odd', 'even', 'odd']
  \end{lstlisting}
\end{frame}

% ---------------------------
\renewcommand{\currenttitle}{Higher-order functions}
\begin{frame}[fragile]{\currenttitle}
  \texttt{map} is a \textbf{higher-order function} because it takes a function
  as an argument \\[2em]

  Let's look at some other core FP higher-order functions.
\end{frame}

% ---------------------------
\renewcommand{\currenttitle}{\texttt{filter}}
\begin{frame}[fragile]{\currenttitle}
  \texttt{filter} effectively removes values that don't pass a test (the
  \textbf{predicate}):

  \begin{lstlisting}[language=Python]
def is_even(n):
  return n % 2 == 0

iter = iterator([1, 2, 3, 4, 5, 6, 7])
evens = filter(is_even, iter)

print(list(evens))
# [2, 4, 6]
  \end{lstlisting}
\end{frame}

% ---------------------------
\renewcommand{\currenttitle}{\texttt{flatmap}}
\begin{frame}[fragile]{\currenttitle}
  \onslide<1->{
    \texttt{flatmap} is similar to map, but the given function should return a
    list. The list of lists is \textit{flattened} into one list. \\[1em]
  }

  \begin{lstlisting}[language=Python]
iter = iterator(['a', 'b', 'c'])

# We're using Python's lambda (anonymous function) syntax
nested = map(lambda c: [c] * 3, iter)
print(list(nested))
# [ ['a', 'a', 'a'], ['b', 'b', 'b'], ['c', 'c', 'c'] ]

flattened = flatmap(lambda c: [c] * 3, iter)
print(list(flattened))
# ['a', 'a', 'a', 'b', 'b', 'b', 'c', 'c', 'c']
  \end{lstlisting}
\end{frame}

% ---------------------------
\renewcommand{\currenttitle}{\texttt{fold} / \texttt{reduce}}
\begin{frame}[fragile]{\currenttitle}
  \texttt{fold} or \texttt{reduce} (different names are used in different
  languages / libraries) turns a list into a single value: \\[1em]

  \begin{lstlisting}[language=Python]
def sum(acc, n):
  return acc + n

iter = iterator([1, 2, 3, 4, 5])
sum_val = reduce(sum, iter)

print(sum_val)
# 15
  \end{lstlisting}
\end{frame}

% ---------------------------
\begin{frame}[fragile]{\currenttitle}
  \vspace*{1em}
  \begin{lstlisting}[language=Python]
def sum(acc, n):
  # acc : accumulated value
  # n   : current value 
  # return value will be `acc` for the next iteration
  return acc + n

iter = iterator([1, 2, 3, 4, 5])
sum_val = reduce(sum, iter)
# acc is initially the first value (1)
# 1 + 2 = 3
# 3 + 3 = 6
# 6 + 4 = 10
# 10 + 5 = 15

print(sum_val)
# 15
  \end{lstlisting}
\end{frame}

% ---------------------------
\renewcommand{\currenttitle}{Other iterator functions}
\begin{frame}[fragile]{\currenttitle}
  Many languages / libraries have many other utilities besides \texttt{map},
  \texttt{flatmap} and \texttt{reduce}.

  Here are some from the Rust standard library: \\[3em]

  \begin{center}
    \begin{tabular}{ r l }
      \texttt{chain(iter)}      & follow an iterator by the values of another   \\
      \texttt{filter\_map(fn)}  & map and filter together                       \\
      \texttt{sum()}            & sum the values                                \\
      \texttt{skip(n)}          & skip (and drop) the first \texttt{n} values   \\
      \texttt{take(n)}          & yield the first \texttt{n} values             \\
      \texttt{take\_while(fn)}  & yield values until \texttt{fn} returns false  \\
    \end{tabular}
  \end{center}

\end{frame}

% ---------------------------
\renewcommand{\currenttitle}{Composing iterator functions}
\begin{frame}[fragile]{\currenttitle}
  Iterator functions are easy to compose and read (once you get used to them).

  Given that \texttt{f1}, \texttt{f2}, \texttt{f3} are functions of appropriate
  types, we can:

  \begin{lstlisting}[language=Python]
  iter = iterator(some\_list)
  reduce(f1, filter(f2, flatmap(f3, take(5, iter))))
  \end{lstlisting}

  In some languages, something like this is equivalent:

  \begin{lstlisting}[language=Python]
  iter.take(5).flatmap(f3).filter(f2).reduce(f1)
  \end{lstlisting}

  \onslide<2->{
    Let's see \texttt{flatmap}, \texttt{filter}, and \texttt{reduce} together in
    one example.
  }
\end{frame}

% ---------------------------
\begin{frame}[fragile]{\currenttitle}
For a list of numbers, compute the product of the positive factors of the
numbers: \\[1em]

  \begin{lstlisting}[language=Python]
def factors(n):
  def is_factor(x):
    return x % n == 0
  return filter(is_factor, range(1, n // 2 + 1))

iter = iterator([2, 4, 8, 12])
product = reduce(lambda acc, n: acc * n,
                 flatmap(factors, iter))
print(product)
# 1769472 = 2 * 2 * 4 * 2 * 4 * 8 * 2 * 3 * 4 * 6 * 12
# 1's were omitted for brevity

  \end{lstlisting}
\end{frame}

% ---------------------------
\renewcommand{\currenttitle}{How can I use it?}
\begin{frame}[fragile]{\currenttitle}
  These demonstrations are all fine and good

  But how can we use these ideas in real code?
\end{frame}

% ---------------------------
\renewcommand{\currenttitle}{In Python}
\begin{frame}[fragile]{\currenttitle}
  Python has built-in \texttt{map} and \texttt{filter} functions almost
  identical to the demonstration ones:

  \begin{lstlisting}[language=Python]
# You can pass in lists directly, because they're
# considered iterators.
numbers = [-10, -5, 0, 5, 10]
filtered = map(lambda n: n + n,
               filter(lambda n: n >= 0, numbers))

# We still have to collect the iterator to a list using
# the `list` function.
print(list(filtered))
# [0, 10, 20]
  \end{lstlisting}
\end{frame}

% ---------------------------
\renewcommand{\currenttitle}{In Python: \texttt{itertools}}
\begin{frame}[fragile]{\currenttitle}
  Other higher-order functions can be found in the \texttt{itertools} module:

  \begin{lstlisting}[language=Python]
from itertools import islice, chain
# islice  : take first `n` values 
# chain   : chain iterators sequentially

# Get the first five values from 3 iterators
list0 = [1]
list1 = [5, 6]
list2 = [99, 98, 97, 96, 95, 94]

first_five = islice(chain(list0, list1, list2), 5)
print(list(first_five))
# [1, 5, 6, 99, 98]
  \end{lstlisting}
\end{frame}

% ---------------------------
\begin{frame}[fragile]{\currenttitle}
  We can emulate \texttt{flatmap} like so: \\[1em]

  \begin{lstlisting}[language=Python]
from itertools import chain

def flatmap(f, iter):
  return chain.from_iterable(map(f, iter))
  \end{lstlisting}
\end{frame}

% ---------------------------
\renewcommand{\currenttitle}{In Python: List comprehensions}
\begin{frame}[fragile]{\currenttitle}
  Python has something nifty borrowed from FP languages called \textbf{list
  comprehensions}.

  The list comprehension equivalent of our original \texttt{map} example:

  \begin{lstlisting}[language=Python]
def square(n):
  return n * n

numbers = [0, 1, 2, 3, 4, 5]
new_numbers = [square(n) for n in numbers]

# Or:
new_numbers = [n * n for n in numbers]
  \end{lstlisting}
\end{frame}

% ---------------------------
\begin{frame}[fragile]{\currenttitle}
  We can also emulate \texttt{filter} and \texttt{flatmap} using list
  comprehensions:

  \begin{lstlisting}[language=Python]
numbers = [0, 1, 2, 3, 4, 5]

# `filter`
filtered = [n for n in list if n >= 3]
# [3, 4, 5]

# `flatmap`, but this is pretty hard to read
flatmapped = [x for n in list for x in [n] * 3]
# [0, 0, 0, 1, 1, 1, 2, 2, 2, ...]
  \end{lstlisting}
\end{frame}

% ---------------------------
\renewcommand{\currenttitle}{In Java}
\begin{frame}[fragile]{\currenttitle}
  Java 8 added `Streams`, which are basically iterators:

  \begin{lstlisting}[language=Java]
import java.util.Arrays;
import java.util.stream.Collectors;

int[] numbers = new int[] { 1, 2, 3, 4, 5 };
List<Integer> squared = Arrays.stream(numbers)
                              .map(n -> n * n)
                              .collect(Collectors.toList());
# { 1, 4, 9, 16, 25 }
  \end{lstlisting}
\end{frame}

% ---------------------------
\begin{frame}[fragile]{\currenttitle}
  We can also call the \texttt{stream} method on \texttt{List}'s,
  \texttt{HashMap}'s, etc:

  \begin{lstlisting}[language=Java]
List<String> list = new ArrayList<>();
list.add("Java");
list.add("is");
list.add("boring");
list.add("!");

List<String> big = list.stream()
                       .map(s -> s.toUpperCase())
                       .collect(Collectors.toList());
# { "JAVA", "IS", "BORING", "!" }
  \end{lstlisting}

  Check out the \texttt{Stream} docs for all the iterator functions.
\end{frame}

% ---------------------------
\renewcommand{\currenttitle}{Let's try}
\begin{frame}[fragile]{\currenttitle}
  Given a list \texttt{list}, compute the sum of the first five numbers that
  are divisible by 5:

  \begin{lstlisting}[language=Python]
# Python
def selectively_stringify(list):
  # Your implementation here
  \end{lstlisting}

  \begin{lstlisting}[language=Java]
// Java
int selectivelyStringify(List<Integer> list) {
  // Your implementation here
}
  \end{lstlisting}

  Try writing an implementation with a \texttt{for} loop first, then one with
  iterators.

\end{frame}

% ---------------------------
\renewcommand{\currenttitle}{Let's try: Python solutions}
\begin{frame}[fragile]{\currenttitle}
  \begin{lstlisting}[language=Python]
# Python: for loop solution
def selectively_stringify(list):
  c = 0
  sum = 0
  for n in list:
    if c < 5:
      if n % 5 == 0:
        sum += n
        c += 1
    else:
      break
  return sum
  \end{lstlisting}
\end{frame}

% ---------------------------
\begin{frame}[fragile]{\currenttitle}
  \begin{lstlisting}[language=Python]
# Python
def selectively_stringify(list):
  def divisible(n):
    return n % 5 == 0

  # List comprehension solution:
  return sum([n for n in list if divisible(n)][:5])
  \end{lstlisting}
\end{frame}

% ---------------------------
\begin{frame}[fragile]{\currenttitle}
  \begin{lstlisting}[language=Python]
# Faster, better implementation:
iter = filter(lambda n: divisible(n), list)
return sum([next(iter) for _ in range(5)])


# Or, using itertools:
from itertools import islice

iter = filter(lambda n: divisible(n), list)
return sum(islice(iter, 5))
  \end{lstlisting}
\end{frame}

% ---------------------------
\renewcommand{\currenttitle}{Let's try: Java solutions}
\begin{frame}[fragile]{\currenttitle}
  \begin{lstlisting}[language=Java]
// Java: for loop solution
int selectivelyStringify(List<Integer> list) {
  int c = 0;
  int sum = 0;
  for (int n : list) {
    if c < 5 {
      if n % 5 == 0 {
        sum += n;
        c += 1;
      }
    } else { break; }
  }
  return sum;
}
  \end{lstlisting}
\end{frame}

% ---------------------------
\begin{frame}[fragile]{\currenttitle}
  \begin{lstlisting}[language=Java]
// Java: iterator solution
int selectivelyStringify(List<Integer> list) {
  return list.stream()
             .filter(divisible)
             .limit(5)
             .sum();
}

boolean divisible(int n) {
  return n % 5 == 0
}
  \end{lstlisting}
\end{frame}


% ---------------------------
\renewcommand{\currenttitle}{Try another}
\begin{frame}[fragile]{\currenttitle}
  Given a sentence, compute the product of the lengths of each word in the
  sentence, but only if the word is not "the":

  \begin{lstlisting}[language=Python]
  # Python
  def words_not_the(sentence):
    # Your implementation here!
  \end{lstlisting}

  \begin{lstlisting}[language=Java]
  // Java
  int wordsNotThe(String sentence) {
    // Your implementation here!
  }
  \end{lstlisting}
\end{frame}

% ---------------------------
\renewcommand{\currenttitle}{Try another: Python solution}
\begin{frame}[fragile]{\currenttitle}
  A list comprehension solution:

  \begin{lstlisting}[language=Python]
from math import prd

# List comprehension
def words_not_the(sentence):
  return prod([len(w) for w in sentence.split(' ')
               if w != 'the'])
  \end{lstlisting}
\end{frame}

% ---------------------------
\begin{frame}[fragile]{\currenttitle}
  An iterator solution:

  \begin{lstlisting}[language=Python]
from math import prd

# Iterator solution
def words_not_the(sentence):
  return prod(map(len, filter(lambda w: w != 'the',
                              sentence.split(' '))))

# We can of course not write a one-liner:
def words_not_the(sentence):
  words = sentence.split(' ')
  filtered = filter(lambda w: w != 'the', words)
  lengths = map(len, filtered)
  return prod(lengths)
  \end{lstlisting}
\end{frame}

% ---------------------------
\renewcommand{\currenttitle}{Try another: Java solution}
\begin{frame}[fragile]{\currenttitle}
  A stream solution:

  \begin{lstlisting}[language=Java]
import java.util.Arrays;

int wordsNotThe(String sentence) {
  return Arrays.stream(words.split(" "))
               .filter(w -> !w.equals("the"))
               .map(w -> w.length())
               .product();
}
  \end{lstlisting}
\end{frame}

% ---------------------------
\renewcommand{\currenttitle}{What next?}
\begin{frame}{\currenttitle}
  If you're interested in making your code more concise and clear with
  iterators: \\[1em]

  \begin{itemize}
    \item Learn about implementing iterator classes in Python
    \item Learn about Generators in Python (and in general)
    \item Become more familiar with Java's \texttt{Stream} API
    \item Learn FP core concepts (purity, immutability, lazy eval, recursion, etc.)
    \item Actually learn a functional programming language (Haskell, Lisp, etc.)
  \end{itemize}
\end{frame}

% ---------------------------
\renewcommand{\currenttitle}{Thanks!}
\begin{frame}[fragile]{\currenttitle}
  Slides here: \url{https://github.com/stogacs/basicfp}
\end{frame}

\end{document}
